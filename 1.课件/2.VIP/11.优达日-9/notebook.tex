
% Default to the notebook output style

    


% Inherit from the specified cell style.




    
\documentclass[11pt]{article}

    
    
    \usepackage[T1]{fontenc}
    % Nicer default font (+ math font) than Computer Modern for most use cases
    \usepackage{mathpazo}

    % Basic figure setup, for now with no caption control since it's done
    % automatically by Pandoc (which extracts ![](path) syntax from Markdown).
    \usepackage{graphicx}
    % We will generate all images so they have a width \maxwidth. This means
    % that they will get their normal width if they fit onto the page, but
    % are scaled down if they would overflow the margins.
    \makeatletter
    \def\maxwidth{\ifdim\Gin@nat@width>\linewidth\linewidth
    \else\Gin@nat@width\fi}
    \makeatother
    \let\Oldincludegraphics\includegraphics
    % Set max figure width to be 80% of text width, for now hardcoded.
    \renewcommand{\includegraphics}[1]{\Oldincludegraphics[width=.8\maxwidth]{#1}}
    % Ensure that by default, figures have no caption (until we provide a
    % proper Figure object with a Caption API and a way to capture that
    % in the conversion process - todo).
    \usepackage{caption}
    \DeclareCaptionLabelFormat{nolabel}{}
    \captionsetup{labelformat=nolabel}

    \usepackage{adjustbox} % Used to constrain images to a maximum size 
    \usepackage{xcolor} % Allow colors to be defined
    \usepackage{enumerate} % Needed for markdown enumerations to work
    \usepackage{geometry} % Used to adjust the document margins
    \usepackage{amsmath} % Equations
    \usepackage{amssymb} % Equations
    \usepackage{textcomp} % defines textquotesingle
    % Hack from http://tex.stackexchange.com/a/47451/13684:
    \AtBeginDocument{%
        \def\PYZsq{\textquotesingle}% Upright quotes in Pygmentized code
    }
    \usepackage{upquote} % Upright quotes for verbatim code
    \usepackage{eurosym} % defines \euro
    \usepackage[mathletters]{ucs} % Extended unicode (utf-8) support
    \usepackage[utf8x]{inputenc} % Allow utf-8 characters in the tex document
    \usepackage{fancyvrb} % verbatim replacement that allows latex
    \usepackage{grffile} % extends the file name processing of package graphics 
                         % to support a larger range 
    % The hyperref package gives us a pdf with properly built
    % internal navigation ('pdf bookmarks' for the table of contents,
    % internal cross-reference links, web links for URLs, etc.)
    \usepackage{hyperref}
    \usepackage{longtable} % longtable support required by pandoc >1.10
    \usepackage{booktabs}  % table support for pandoc > 1.12.2
    \usepackage[inline]{enumitem} % IRkernel/repr support (it uses the enumerate* environment)
    \usepackage[normalem]{ulem} % ulem is needed to support strikethroughs (\sout)
                                % normalem makes italics be italics, not underlines
    

    
    
    % Colors for the hyperref package
    \definecolor{urlcolor}{rgb}{0,.145,.698}
    \definecolor{linkcolor}{rgb}{.71,0.21,0.01}
    \definecolor{citecolor}{rgb}{.12,.54,.11}

    % ANSI colors
    \definecolor{ansi-black}{HTML}{3E424D}
    \definecolor{ansi-black-intense}{HTML}{282C36}
    \definecolor{ansi-red}{HTML}{E75C58}
    \definecolor{ansi-red-intense}{HTML}{B22B31}
    \definecolor{ansi-green}{HTML}{00A250}
    \definecolor{ansi-green-intense}{HTML}{007427}
    \definecolor{ansi-yellow}{HTML}{DDB62B}
    \definecolor{ansi-yellow-intense}{HTML}{B27D12}
    \definecolor{ansi-blue}{HTML}{208FFB}
    \definecolor{ansi-blue-intense}{HTML}{0065CA}
    \definecolor{ansi-magenta}{HTML}{D160C4}
    \definecolor{ansi-magenta-intense}{HTML}{A03196}
    \definecolor{ansi-cyan}{HTML}{60C6C8}
    \definecolor{ansi-cyan-intense}{HTML}{258F8F}
    \definecolor{ansi-white}{HTML}{C5C1B4}
    \definecolor{ansi-white-intense}{HTML}{A1A6B2}

    % commands and environments needed by pandoc snippets
    % extracted from the output of `pandoc -s`
    \providecommand{\tightlist}{%
      \setlength{\itemsep}{0pt}\setlength{\parskip}{0pt}}
    \DefineVerbatimEnvironment{Highlighting}{Verbatim}{commandchars=\\\{\}}
    % Add ',fontsize=\small' for more characters per line
    \newenvironment{Shaded}{}{}
    \newcommand{\KeywordTok}[1]{\textcolor[rgb]{0.00,0.44,0.13}{\textbf{{#1}}}}
    \newcommand{\DataTypeTok}[1]{\textcolor[rgb]{0.56,0.13,0.00}{{#1}}}
    \newcommand{\DecValTok}[1]{\textcolor[rgb]{0.25,0.63,0.44}{{#1}}}
    \newcommand{\BaseNTok}[1]{\textcolor[rgb]{0.25,0.63,0.44}{{#1}}}
    \newcommand{\FloatTok}[1]{\textcolor[rgb]{0.25,0.63,0.44}{{#1}}}
    \newcommand{\CharTok}[1]{\textcolor[rgb]{0.25,0.44,0.63}{{#1}}}
    \newcommand{\StringTok}[1]{\textcolor[rgb]{0.25,0.44,0.63}{{#1}}}
    \newcommand{\CommentTok}[1]{\textcolor[rgb]{0.38,0.63,0.69}{\textit{{#1}}}}
    \newcommand{\OtherTok}[1]{\textcolor[rgb]{0.00,0.44,0.13}{{#1}}}
    \newcommand{\AlertTok}[1]{\textcolor[rgb]{1.00,0.00,0.00}{\textbf{{#1}}}}
    \newcommand{\FunctionTok}[1]{\textcolor[rgb]{0.02,0.16,0.49}{{#1}}}
    \newcommand{\RegionMarkerTok}[1]{{#1}}
    \newcommand{\ErrorTok}[1]{\textcolor[rgb]{1.00,0.00,0.00}{\textbf{{#1}}}}
    \newcommand{\NormalTok}[1]{{#1}}
    
    % Additional commands for more recent versions of Pandoc
    \newcommand{\ConstantTok}[1]{\textcolor[rgb]{0.53,0.00,0.00}{{#1}}}
    \newcommand{\SpecialCharTok}[1]{\textcolor[rgb]{0.25,0.44,0.63}{{#1}}}
    \newcommand{\VerbatimStringTok}[1]{\textcolor[rgb]{0.25,0.44,0.63}{{#1}}}
    \newcommand{\SpecialStringTok}[1]{\textcolor[rgb]{0.73,0.40,0.53}{{#1}}}
    \newcommand{\ImportTok}[1]{{#1}}
    \newcommand{\DocumentationTok}[1]{\textcolor[rgb]{0.73,0.13,0.13}{\textit{{#1}}}}
    \newcommand{\AnnotationTok}[1]{\textcolor[rgb]{0.38,0.63,0.69}{\textbf{\textit{{#1}}}}}
    \newcommand{\CommentVarTok}[1]{\textcolor[rgb]{0.38,0.63,0.69}{\textbf{\textit{{#1}}}}}
    \newcommand{\VariableTok}[1]{\textcolor[rgb]{0.10,0.09,0.49}{{#1}}}
    \newcommand{\ControlFlowTok}[1]{\textcolor[rgb]{0.00,0.44,0.13}{\textbf{{#1}}}}
    \newcommand{\OperatorTok}[1]{\textcolor[rgb]{0.40,0.40,0.40}{{#1}}}
    \newcommand{\BuiltInTok}[1]{{#1}}
    \newcommand{\ExtensionTok}[1]{{#1}}
    \newcommand{\PreprocessorTok}[1]{\textcolor[rgb]{0.74,0.48,0.00}{{#1}}}
    \newcommand{\AttributeTok}[1]{\textcolor[rgb]{0.49,0.56,0.16}{{#1}}}
    \newcommand{\InformationTok}[1]{\textcolor[rgb]{0.38,0.63,0.69}{\textbf{\textit{{#1}}}}}
    \newcommand{\WarningTok}[1]{\textcolor[rgb]{0.38,0.63,0.69}{\textbf{\textit{{#1}}}}}
    
    
    % Define a nice break command that doesn't care if a line doesn't already
    % exist.
    \def\br{\hspace*{\fill} \\* }
    % Math Jax compatability definitions
    \def\gt{>}
    \def\lt{<}
    % Document parameters
    \title{perceptron}
    
    
    

    % Pygments definitions
    
\makeatletter
\def\PY@reset{\let\PY@it=\relax \let\PY@bf=\relax%
    \let\PY@ul=\relax \let\PY@tc=\relax%
    \let\PY@bc=\relax \let\PY@ff=\relax}
\def\PY@tok#1{\csname PY@tok@#1\endcsname}
\def\PY@toks#1+{\ifx\relax#1\empty\else%
    \PY@tok{#1}\expandafter\PY@toks\fi}
\def\PY@do#1{\PY@bc{\PY@tc{\PY@ul{%
    \PY@it{\PY@bf{\PY@ff{#1}}}}}}}
\def\PY#1#2{\PY@reset\PY@toks#1+\relax+\PY@do{#2}}

\expandafter\def\csname PY@tok@w\endcsname{\def\PY@tc##1{\textcolor[rgb]{0.73,0.73,0.73}{##1}}}
\expandafter\def\csname PY@tok@c\endcsname{\let\PY@it=\textit\def\PY@tc##1{\textcolor[rgb]{0.25,0.50,0.50}{##1}}}
\expandafter\def\csname PY@tok@cp\endcsname{\def\PY@tc##1{\textcolor[rgb]{0.74,0.48,0.00}{##1}}}
\expandafter\def\csname PY@tok@k\endcsname{\let\PY@bf=\textbf\def\PY@tc##1{\textcolor[rgb]{0.00,0.50,0.00}{##1}}}
\expandafter\def\csname PY@tok@kp\endcsname{\def\PY@tc##1{\textcolor[rgb]{0.00,0.50,0.00}{##1}}}
\expandafter\def\csname PY@tok@kt\endcsname{\def\PY@tc##1{\textcolor[rgb]{0.69,0.00,0.25}{##1}}}
\expandafter\def\csname PY@tok@o\endcsname{\def\PY@tc##1{\textcolor[rgb]{0.40,0.40,0.40}{##1}}}
\expandafter\def\csname PY@tok@ow\endcsname{\let\PY@bf=\textbf\def\PY@tc##1{\textcolor[rgb]{0.67,0.13,1.00}{##1}}}
\expandafter\def\csname PY@tok@nb\endcsname{\def\PY@tc##1{\textcolor[rgb]{0.00,0.50,0.00}{##1}}}
\expandafter\def\csname PY@tok@nf\endcsname{\def\PY@tc##1{\textcolor[rgb]{0.00,0.00,1.00}{##1}}}
\expandafter\def\csname PY@tok@nc\endcsname{\let\PY@bf=\textbf\def\PY@tc##1{\textcolor[rgb]{0.00,0.00,1.00}{##1}}}
\expandafter\def\csname PY@tok@nn\endcsname{\let\PY@bf=\textbf\def\PY@tc##1{\textcolor[rgb]{0.00,0.00,1.00}{##1}}}
\expandafter\def\csname PY@tok@ne\endcsname{\let\PY@bf=\textbf\def\PY@tc##1{\textcolor[rgb]{0.82,0.25,0.23}{##1}}}
\expandafter\def\csname PY@tok@nv\endcsname{\def\PY@tc##1{\textcolor[rgb]{0.10,0.09,0.49}{##1}}}
\expandafter\def\csname PY@tok@no\endcsname{\def\PY@tc##1{\textcolor[rgb]{0.53,0.00,0.00}{##1}}}
\expandafter\def\csname PY@tok@nl\endcsname{\def\PY@tc##1{\textcolor[rgb]{0.63,0.63,0.00}{##1}}}
\expandafter\def\csname PY@tok@ni\endcsname{\let\PY@bf=\textbf\def\PY@tc##1{\textcolor[rgb]{0.60,0.60,0.60}{##1}}}
\expandafter\def\csname PY@tok@na\endcsname{\def\PY@tc##1{\textcolor[rgb]{0.49,0.56,0.16}{##1}}}
\expandafter\def\csname PY@tok@nt\endcsname{\let\PY@bf=\textbf\def\PY@tc##1{\textcolor[rgb]{0.00,0.50,0.00}{##1}}}
\expandafter\def\csname PY@tok@nd\endcsname{\def\PY@tc##1{\textcolor[rgb]{0.67,0.13,1.00}{##1}}}
\expandafter\def\csname PY@tok@s\endcsname{\def\PY@tc##1{\textcolor[rgb]{0.73,0.13,0.13}{##1}}}
\expandafter\def\csname PY@tok@sd\endcsname{\let\PY@it=\textit\def\PY@tc##1{\textcolor[rgb]{0.73,0.13,0.13}{##1}}}
\expandafter\def\csname PY@tok@si\endcsname{\let\PY@bf=\textbf\def\PY@tc##1{\textcolor[rgb]{0.73,0.40,0.53}{##1}}}
\expandafter\def\csname PY@tok@se\endcsname{\let\PY@bf=\textbf\def\PY@tc##1{\textcolor[rgb]{0.73,0.40,0.13}{##1}}}
\expandafter\def\csname PY@tok@sr\endcsname{\def\PY@tc##1{\textcolor[rgb]{0.73,0.40,0.53}{##1}}}
\expandafter\def\csname PY@tok@ss\endcsname{\def\PY@tc##1{\textcolor[rgb]{0.10,0.09,0.49}{##1}}}
\expandafter\def\csname PY@tok@sx\endcsname{\def\PY@tc##1{\textcolor[rgb]{0.00,0.50,0.00}{##1}}}
\expandafter\def\csname PY@tok@m\endcsname{\def\PY@tc##1{\textcolor[rgb]{0.40,0.40,0.40}{##1}}}
\expandafter\def\csname PY@tok@gh\endcsname{\let\PY@bf=\textbf\def\PY@tc##1{\textcolor[rgb]{0.00,0.00,0.50}{##1}}}
\expandafter\def\csname PY@tok@gu\endcsname{\let\PY@bf=\textbf\def\PY@tc##1{\textcolor[rgb]{0.50,0.00,0.50}{##1}}}
\expandafter\def\csname PY@tok@gd\endcsname{\def\PY@tc##1{\textcolor[rgb]{0.63,0.00,0.00}{##1}}}
\expandafter\def\csname PY@tok@gi\endcsname{\def\PY@tc##1{\textcolor[rgb]{0.00,0.63,0.00}{##1}}}
\expandafter\def\csname PY@tok@gr\endcsname{\def\PY@tc##1{\textcolor[rgb]{1.00,0.00,0.00}{##1}}}
\expandafter\def\csname PY@tok@ge\endcsname{\let\PY@it=\textit}
\expandafter\def\csname PY@tok@gs\endcsname{\let\PY@bf=\textbf}
\expandafter\def\csname PY@tok@gp\endcsname{\let\PY@bf=\textbf\def\PY@tc##1{\textcolor[rgb]{0.00,0.00,0.50}{##1}}}
\expandafter\def\csname PY@tok@go\endcsname{\def\PY@tc##1{\textcolor[rgb]{0.53,0.53,0.53}{##1}}}
\expandafter\def\csname PY@tok@gt\endcsname{\def\PY@tc##1{\textcolor[rgb]{0.00,0.27,0.87}{##1}}}
\expandafter\def\csname PY@tok@err\endcsname{\def\PY@bc##1{\setlength{\fboxsep}{0pt}\fcolorbox[rgb]{1.00,0.00,0.00}{1,1,1}{\strut ##1}}}
\expandafter\def\csname PY@tok@kc\endcsname{\let\PY@bf=\textbf\def\PY@tc##1{\textcolor[rgb]{0.00,0.50,0.00}{##1}}}
\expandafter\def\csname PY@tok@kd\endcsname{\let\PY@bf=\textbf\def\PY@tc##1{\textcolor[rgb]{0.00,0.50,0.00}{##1}}}
\expandafter\def\csname PY@tok@kn\endcsname{\let\PY@bf=\textbf\def\PY@tc##1{\textcolor[rgb]{0.00,0.50,0.00}{##1}}}
\expandafter\def\csname PY@tok@kr\endcsname{\let\PY@bf=\textbf\def\PY@tc##1{\textcolor[rgb]{0.00,0.50,0.00}{##1}}}
\expandafter\def\csname PY@tok@bp\endcsname{\def\PY@tc##1{\textcolor[rgb]{0.00,0.50,0.00}{##1}}}
\expandafter\def\csname PY@tok@fm\endcsname{\def\PY@tc##1{\textcolor[rgb]{0.00,0.00,1.00}{##1}}}
\expandafter\def\csname PY@tok@vc\endcsname{\def\PY@tc##1{\textcolor[rgb]{0.10,0.09,0.49}{##1}}}
\expandafter\def\csname PY@tok@vg\endcsname{\def\PY@tc##1{\textcolor[rgb]{0.10,0.09,0.49}{##1}}}
\expandafter\def\csname PY@tok@vi\endcsname{\def\PY@tc##1{\textcolor[rgb]{0.10,0.09,0.49}{##1}}}
\expandafter\def\csname PY@tok@vm\endcsname{\def\PY@tc##1{\textcolor[rgb]{0.10,0.09,0.49}{##1}}}
\expandafter\def\csname PY@tok@sa\endcsname{\def\PY@tc##1{\textcolor[rgb]{0.73,0.13,0.13}{##1}}}
\expandafter\def\csname PY@tok@sb\endcsname{\def\PY@tc##1{\textcolor[rgb]{0.73,0.13,0.13}{##1}}}
\expandafter\def\csname PY@tok@sc\endcsname{\def\PY@tc##1{\textcolor[rgb]{0.73,0.13,0.13}{##1}}}
\expandafter\def\csname PY@tok@dl\endcsname{\def\PY@tc##1{\textcolor[rgb]{0.73,0.13,0.13}{##1}}}
\expandafter\def\csname PY@tok@s2\endcsname{\def\PY@tc##1{\textcolor[rgb]{0.73,0.13,0.13}{##1}}}
\expandafter\def\csname PY@tok@sh\endcsname{\def\PY@tc##1{\textcolor[rgb]{0.73,0.13,0.13}{##1}}}
\expandafter\def\csname PY@tok@s1\endcsname{\def\PY@tc##1{\textcolor[rgb]{0.73,0.13,0.13}{##1}}}
\expandafter\def\csname PY@tok@mb\endcsname{\def\PY@tc##1{\textcolor[rgb]{0.40,0.40,0.40}{##1}}}
\expandafter\def\csname PY@tok@mf\endcsname{\def\PY@tc##1{\textcolor[rgb]{0.40,0.40,0.40}{##1}}}
\expandafter\def\csname PY@tok@mh\endcsname{\def\PY@tc##1{\textcolor[rgb]{0.40,0.40,0.40}{##1}}}
\expandafter\def\csname PY@tok@mi\endcsname{\def\PY@tc##1{\textcolor[rgb]{0.40,0.40,0.40}{##1}}}
\expandafter\def\csname PY@tok@il\endcsname{\def\PY@tc##1{\textcolor[rgb]{0.40,0.40,0.40}{##1}}}
\expandafter\def\csname PY@tok@mo\endcsname{\def\PY@tc##1{\textcolor[rgb]{0.40,0.40,0.40}{##1}}}
\expandafter\def\csname PY@tok@ch\endcsname{\let\PY@it=\textit\def\PY@tc##1{\textcolor[rgb]{0.25,0.50,0.50}{##1}}}
\expandafter\def\csname PY@tok@cm\endcsname{\let\PY@it=\textit\def\PY@tc##1{\textcolor[rgb]{0.25,0.50,0.50}{##1}}}
\expandafter\def\csname PY@tok@cpf\endcsname{\let\PY@it=\textit\def\PY@tc##1{\textcolor[rgb]{0.25,0.50,0.50}{##1}}}
\expandafter\def\csname PY@tok@c1\endcsname{\let\PY@it=\textit\def\PY@tc##1{\textcolor[rgb]{0.25,0.50,0.50}{##1}}}
\expandafter\def\csname PY@tok@cs\endcsname{\let\PY@it=\textit\def\PY@tc##1{\textcolor[rgb]{0.25,0.50,0.50}{##1}}}

\def\PYZbs{\char`\\}
\def\PYZus{\char`\_}
\def\PYZob{\char`\{}
\def\PYZcb{\char`\}}
\def\PYZca{\char`\^}
\def\PYZam{\char`\&}
\def\PYZlt{\char`\<}
\def\PYZgt{\char`\>}
\def\PYZsh{\char`\#}
\def\PYZpc{\char`\%}
\def\PYZdl{\char`\$}
\def\PYZhy{\char`\-}
\def\PYZsq{\char`\'}
\def\PYZdq{\char`\"}
\def\PYZti{\char`\~}
% for compatibility with earlier versions
\def\PYZat{@}
\def\PYZlb{[}
\def\PYZrb{]}
\makeatother


    % Exact colors from NB
    \definecolor{incolor}{rgb}{0.0, 0.0, 0.5}
    \definecolor{outcolor}{rgb}{0.545, 0.0, 0.0}



    
    % Prevent overflowing lines due to hard-to-break entities
    \sloppy 
    % Setup hyperref package
    \hypersetup{
      breaklinks=true,  % so long urls are correctly broken across lines
      colorlinks=true,
      urlcolor=urlcolor,
      linkcolor=linkcolor,
      citecolor=citecolor,
      }
    % Slightly bigger margins than the latex defaults
    
    \geometry{verbose,tmargin=1in,bmargin=1in,lmargin=1in,rmargin=1in}
    
    

    \begin{document}
    
    
    \maketitle
    
    

    
    \hypertarget{ux611fux77e5ux673a}{%
\section{感知机}\label{ux611fux77e5ux673a}}

    \begin{Verbatim}[commandchars=\\\{\}]
{\color{incolor}In [{\color{incolor}1}]:} \PY{k+kn}{import} \PY{n+nn}{numpy} \PY{k}{as} \PY{n+nn}{np}
        \PY{k+kn}{import} \PY{n+nn}{matplotlib}\PY{n+nn}{.}\PY{n+nn}{pyplot} \PY{k}{as} \PY{n+nn}{plt}
        
        \PY{o}{\PYZpc{}}\PY{k}{matplotlib} inline
\end{Verbatim}


    \begin{Verbatim}[commandchars=\\\{\}]
{\color{incolor}In [{\color{incolor}2}]:} \PY{k+kn}{from} \PY{n+nn}{sklearn}\PY{n+nn}{.}\PY{n+nn}{datasets} \PY{k}{import} \PY{n}{make\PYZus{}classification}\PY{p}{,} \PY{n}{make\PYZus{}blobs}
        \PY{k+kn}{from} \PY{n+nn}{sklearn}\PY{n+nn}{.}\PY{n+nn}{linear\PYZus{}model} \PY{k}{import} \PY{n}{LogisticRegression}
        
        \PY{n}{np}\PY{o}{.}\PY{n}{random}\PY{o}{.}\PY{n}{seed}\PY{p}{(}\PY{l+m+mi}{600}\PY{p}{)}
        \PY{c+c1}{\PYZsh{} 生成数据}
        \PY{n}{X}\PY{p}{,} \PY{n}{y} \PY{o}{=} \PY{n}{make\PYZus{}blobs}\PY{p}{(}\PY{n}{n\PYZus{}samples}\PY{o}{=}\PY{l+m+mi}{200}\PY{p}{,}\PY{n}{n\PYZus{}features}\PY{o}{=}\PY{l+m+mi}{2}\PY{p}{,} \PY{n}{centers}\PY{o}{=}\PY{l+m+mi}{2}\PY{p}{)}
        
        \PY{c+c1}{\PYZsh{} 这里创建了一个逻辑回归模型}
        \PY{n}{model} \PY{o}{=} \PY{n}{LogisticRegression}\PY{p}{(}\PY{n}{solver}\PY{o}{=}\PY{l+s+s1}{\PYZsq{}}\PY{l+s+s1}{lbfgs}\PY{l+s+s1}{\PYZsq{}}\PY{p}{)}
        \PY{n}{model}\PY{o}{.}\PY{n}{fit}\PY{p}{(}\PY{n}{X}\PY{p}{,} \PY{n}{y}\PY{p}{)}
        
        \PY{n}{w0}\PY{p}{,} \PY{n}{w1} \PY{o}{=} \PY{n}{model}\PY{o}{.}\PY{n}{coef\PYZus{}}\PY{p}{[}\PY{l+m+mi}{0}\PY{p}{]}
        \PY{n}{b} \PY{o}{=} \PY{n}{model}\PY{o}{.}\PY{n}{intercept\PYZus{}}
        
        \PY{n}{line\PYZus{}x0} \PY{o}{=} \PY{p}{[}\PY{o}{\PYZhy{}}\PY{l+m+mi}{12}\PY{p}{,} \PY{o}{\PYZhy{}}\PY{l+m+mi}{4}\PY{p}{]}
        \PY{n}{line\PYZus{}x1} \PY{o}{=} \PY{p}{[}\PY{p}{(}\PY{o}{\PYZhy{}}\PY{n}{b}\PY{o}{\PYZhy{}}\PY{n}{w0}\PY{o}{*}\PY{p}{(}\PY{o}{\PYZhy{}}\PY{l+m+mi}{12}\PY{p}{)}\PY{p}{)}\PY{o}{/} \PY{n}{w1}\PY{p}{,} \PY{p}{(}\PY{o}{\PYZhy{}}\PY{l+m+mi}{4}\PY{o}{*}\PY{n}{w0}\PY{o}{\PYZhy{}}\PY{n}{b}\PY{p}{)} \PY{o}{/} \PY{n}{w1}\PY{p}{]}
\end{Verbatim}


    

    \begin{Verbatim}[commandchars=\\\{\}]
{\color{incolor}In [{\color{incolor}7}]:} \PY{n}{plt}\PY{o}{.}\PY{n}{plot}\PY{p}{(}\PY{n}{line\PYZus{}x0}\PY{p}{,} \PY{n}{line\PYZus{}x1}\PY{p}{,} \PY{n}{color}\PY{o}{=}\PY{l+s+s1}{\PYZsq{}}\PY{l+s+s1}{r}\PY{l+s+s1}{\PYZsq{}}\PY{p}{,} \PY{n}{linewidth}\PY{o}{=}\PY{l+m+mi}{3}\PY{p}{)}
        \PY{n}{plt}\PY{o}{.}\PY{n}{scatter}\PY{p}{(}\PY{l+m+mi}{0}\PY{p}{,} \PY{o}{\PYZhy{}}\PY{l+m+mi}{5}\PY{p}{,} \PY{n}{color}\PY{o}{=}\PY{l+s+s1}{\PYZsq{}}\PY{l+s+s1}{r}\PY{l+s+s1}{\PYZsq{}}\PY{p}{)}
        \PY{n}{plt}\PY{o}{.}\PY{n}{scatter}\PY{p}{(}\PY{n}{X}\PY{p}{[}\PY{p}{:}\PY{p}{,} \PY{l+m+mi}{0}\PY{p}{]}\PY{p}{,} \PY{n}{X}\PY{p}{[}\PY{p}{:}\PY{p}{,} \PY{l+m+mi}{1}\PY{p}{]}\PY{p}{,} \PY{n}{marker}\PY{o}{=}\PY{l+s+s1}{\PYZsq{}}\PY{l+s+s1}{o}\PY{l+s+s1}{\PYZsq{}}\PY{p}{,} \PY{n}{c}\PY{o}{=}\PY{n}{y}\PY{p}{)}
        \PY{n}{plt}\PY{o}{.}\PY{n}{xlabel}\PY{p}{(}\PY{l+s+s1}{\PYZsq{}}\PY{l+s+s1}{x0}\PY{l+s+s1}{\PYZsq{}}\PY{p}{)}
        \PY{n}{plt}\PY{o}{.}\PY{n}{ylabel}\PY{p}{(}\PY{l+s+s1}{\PYZsq{}}\PY{l+s+s1}{x1}\PY{l+s+s1}{\PYZsq{}}\PY{p}{)}
        \PY{n}{plt}\PY{o}{.}\PY{n}{show}\PY{p}{(}\PY{p}{)}
\end{Verbatim}


    \begin{center}
    \adjustimage{max size={0.9\linewidth}{0.9\paperheight}}{output_4_0.png}
    \end{center}
    { \hspace*{\fill} \\}
    
    \begin{Verbatim}[commandchars=\\\{\}]
{\color{incolor}In [{\color{incolor} }]:} \PY{n}{w0}\PY{o}{*}\PY{n}{x0} \PY{o}{+} \PY{n}{w1}\PY{o}{*}\PY{n}{x1} \PY{o}{+} \PY{n}{b} \PY{o}{=} \PY{l+m+mi}{0}
\end{Verbatim}


    \begin{Verbatim}[commandchars=\\\{\}]
{\color{incolor}In [{\color{incolor}10}]:} \PY{n+nb}{len}\PY{p}{(}\PY{n}{X}\PY{p}{)}
\end{Verbatim}


\begin{Verbatim}[commandchars=\\\{\}]
{\color{outcolor}Out[{\color{outcolor}10}]:} 200
\end{Verbatim}
            
    \begin{Verbatim}[commandchars=\\\{\}]
{\color{incolor}In [{\color{incolor}11}]:} \PY{n}{X}\PY{p}{[}\PY{p}{:}\PY{l+m+mi}{5}\PY{p}{]}
\end{Verbatim}


\begin{Verbatim}[commandchars=\\\{\}]
{\color{outcolor}Out[{\color{outcolor}11}]:} array([[  7.28678297,  -8.51538322],
                [-10.62425645,   0.33021018],
                [  5.41530338,  -8.30187345],
                [  7.32179896,  -7.04656047],
                [ -7.72847208,  -0.39799556]])
\end{Verbatim}
            
    \begin{Verbatim}[commandchars=\\\{\}]
{\color{incolor}In [{\color{incolor}12}]:} \PY{n+nb}{len}\PY{p}{(}\PY{n}{y}\PY{p}{)} \PY{o}{==} \PY{n+nb}{len}\PY{p}{(}\PY{n}{X}\PY{p}{)}
\end{Verbatim}


\begin{Verbatim}[commandchars=\\\{\}]
{\color{outcolor}Out[{\color{outcolor}12}]:} True
\end{Verbatim}
            
    \begin{Verbatim}[commandchars=\\\{\}]
{\color{incolor}In [{\color{incolor}14}]:} \PY{n}{y}\PY{p}{[}\PY{p}{:}\PY{l+m+mi}{5}\PY{p}{]}
\end{Verbatim}


\begin{Verbatim}[commandchars=\\\{\}]
{\color{outcolor}Out[{\color{outcolor}14}]:} array([1, 0, 1, 1, 0])
\end{Verbatim}
            
    这里,什么是符号函数sign呢?\\
首先了解一下我们的数据,用集合\(T={(\vec x_0, y_0), (\vec x_1, y_1), ..., (\vec x_n, y_n)}\)表示。其中\((\vec x_i, y_i)\),表示数据中的第i个样本,\(x_i\)是样本的特征向量,\(x_i\)的维度是由数据本身决定,或者我们打算采用的特征数决定的,\(y_i\)是样本对于的分类标签。

那么以二维空间为例,即:\(x_i\)是二维的。假设我们现在有一条直线,其表达式是\(w_0 x^{(0)} + w_1 x^{(1)} + b = 0\),其中\(x^{(0)}\),\(x^{(1)}\)代表过该直线的任意一点,\(x^{(0)}\)是横坐标,
\(x^{(1)}\)是纵坐标。\\
现在有一点\((2,5)\),怎么判断他是在直线上方还是下方呢?\\
我们需要把(2,5)带入上面的公式,
\(\hat y = w_0 * 2 + w_1 *5 + b\)。\(\hat y\)是这个线性方程算出来的,代表预测值。如果\(\hat y > 0\)
也就是点在直线上方,这个点被记作属于C1类。如果\(\hat y < 0\)
也就是点在直线上方,这个点被记作属于C2类。

根据上面的解释,由于\(\hat y\)大于0的值最多可能和\(x_i\)的个数一样多,小于0的值也是同样的情况,所以,能不能经过一个函数\(sign = f(\hat y)\),使得\(\hat y\)为变为一个的值,又能反应类别呢。伟大的数学家们就定义了这么一个函数,确实能做到。

\[
sign(\hat y) = 
\begin{cases}
\hat y = 1, \hat y \geq 0\\\\
\hat y = -1, \hat y < 0
\end{cases}
\]

\begin{center}\rule{0.5\linewidth}{\linethickness}\end{center}

\(\hat y\) 经过上面sign函数后,我们就能把\((2,5)\)点分为 +1类 或者 -1类
了。

    那么,我们怎么确定这条直线呢?\\
我们现在已经知道了\(X\),所以就要用这些X来确定这条直线了。这里\(X\)是大写,表示一个矩阵,矩阵的每一行代码我们的一个样本,每一列代表样本的每个特征。

    那么怎么开始求解呢,先随机生成我们的\(\vec w=[w_0, w_1]\) 和
\(b\),这就能在二维空间中确定一条直线了。然后接着看看这条直线能否把这两堆数据分开,如果不能全部分开,那么计算下还差多少(损失函数)才能分开,根据偏差的成调整\(\vec w\)和\(b\)的值。

一直重复这个过程,知道找到合适的\(\vec w\)和\(b\),能把这两堆数据划分开。
计算过程使用梯度下降法,损失函数使用:

\[
L(w, b) = -\sum_{x_i \in M} y_i(\vec w \cdot \vec x_i + b)
\]

    \hypertarget{ux5173ux4e8eux4e3aux4ec0ux4e48ux8981ux4f7fux7528ux68afux5ea6ux4e0bux964dux6cd5ux7684ux4e00ux70b9ux60f3ux6cd5}{%
\subsubsection{关于为什么要使用梯度下降法的一点想法:}\label{ux5173ux4e8eux4e3aux4ec0ux4e48ux8981ux4f7fux7528ux68afux5ea6ux4e0bux964dux6cd5ux7684ux4e00ux70b9ux60f3ux6cd5}}

怎么求\(L(\vec w, b)\)在最小值呢?我们知道数学上,可以使用导数,即\(L^{'}(\vec w, b)=0\)的解。

对L求导数可能出现的情况:对系数矩阵X(因为x已知,我们求解的w和b) \$X =
\textbackslash{}begin\{bmatrix\} -- x\_0 -- \textbackslash{} -- x\_1 --
\textbackslash{} \ldots{}\textbackslash{} -- x\_m -- \textbackslash{}
\textbackslash{}end\{bmatrix\} \$

\(X\)是\(m*n矩阵\),是方程的系数矩阵。同时代表着每个样本的特征向量\(\vec x_i\)是n维的,样本总数是m个。

\begin{enumerate}
\def\labelenumi{\alph{enumi}.}
\tightlist
\item
  如果系数矩阵的秩\(r(X)\)小于增广矩阵的秩\(r(X,-b)\),\(r(XS)<r(X,-b)\),那么方程组无解\(r(X)<r(X,-b)\),那么方程组\(L^{'}(w, b)=0\)无解;\\
\item
  如果系统矩阵的秩小于方程组未知数个数,\(r(X)=r(X,-b)<n\),那么方程组有多个解\(r(X)=r(X,-b)<n\),那么方程组\(L^{'}(w, b)=0\)有多个解。\\
\item
  如果系统矩阵的秩等于方程组未知数个数,\(r(X)=r(X,-b)=n\),那么方程组有唯一解\(r(X)=r(X,-b)=n\),那么方程组\(L^{'}(w, b)=0\)有唯一解。
\end{enumerate}

\begin{enumerate}
\def\labelenumi{\arabic{enumi}.}
\tightlist
\item
  样本数量m\textbf{大于}特征维度n,则以上三种情况都有可能发生
\item
  样本数量m\textbf{小于}特征维度n,则会发生a和b两种情况。
\end{enumerate}

而只有c情况才能用逆求解。\\
所以不能直接使用\(w = -{X}^{-1}{b}\)求解。\\
于是有了梯度下降法求解。

    \hypertarget{ux4e0bux9762ux6765ux89e3ux91caux4e0bux635fux5931ux51fdux6570ux7684ux610fux4e49}{%
\subsubsection{下面来解释下损失函数的意义:}\label{ux4e0bux9762ux6765ux89e3ux91caux4e0bux635fux5931ux51fdux6570ux7684ux610fux4e49}}

对于二维空间来说,那么,二维空间中的任一个点到直线\(w_0 x^{(0)_i} + w_1 x^{(1)_i} + b = 0\)的距离就是:
\[\frac{|w_0 x^{(0)}_i + w_1 x^{(1)}_i + b|}{\sqrt{w_0^2 + w_2^2}}=\frac{1}{||\vec w||} |\vec w \cdot \vec x_i + b|\]

拓展到n维空间,右边的式子也是成立的,表示空间中的任一点到超平面
S\((\vec w \cdot \vec x_i + b)\)的距离。

    下面引用《统计学习》中的话

\begin{center}\rule{0.5\linewidth}{\linethickness}\end{center}

对于未分类的数据(\(\vec x_i, y_i\))来说,\(-y_i(\vec w \cdot \vec x_i + b) > 0\)成立。因为当\(\vec w \cdot \vec x_i + b > 0\)时,\(y_i = -1\);而当\(\vec w \cdot \vec x_i + b < 0\)时,\(y_i = +1\)。所以,误分类点到\(x_i\)到超平面的S的距离为:
\[
-\frac{1}{||\vec w||} |\vec w \cdot \vec x_i + b|
\]

这样,假设超平面S的误分类点集合为M,那么所有误分类点到超平面S的总距离就是:
\[
-\frac{1}{||\vec w||}\sum_{x_i \in M} y_i (\vec w \cdot \vec x_i + b)
\]
不考虑\(\frac{1}{||\vec w||}\),就得到了感知机学习的损失函数,M是误分类点的集合:
\[
L(w, b) = -\sum_{x_i \in M} y_i(\vec w \cdot \vec x_i + b)
\]
因此,如果没有误分类点,损失函数值是0。而且,误分类点越少,误分类点离超平面就越近,损失函数就越小。
所以,我们感知机算法的求解问题就相当度找到使得\(L(\vec w, b)\)最小(这里由于L是大于等于0的,所以L的最小值是0)的\(\vec w、b\)。

    梯度下降法求得是损失函数极小化问题的解,也就是说,求出的解有可能出现局部最优解,二不是全局最优解。

    一下引用书中的步骤:\\
假设误分类点M的集合是固定的,那么损失函数\(L^{'}(\vec w, b)\)的梯度由 \[
\\
\nabla_w L(\vec w, b) = - \sum_{x_i \in M} y_i \vec x_i\\\\\\
\nabla_b L(\vec w, b) = - \sum_{x_i \in M} y_i
\]\\
给出。

随机选取一个误差分类点\((x_i, y_i)\),对\(w, b\)进行更新: \[
\\
\vec w <- \vec w + \eta y_i \vec x_i \\
b <- b + \eta y_i
\]\\
式中\(\eta\)是步长,又称为学习率(learning rate),控制w和b更新的幅度。\\
这样,通过迭代可以期待损失函数\(L^{'}(\vec w, b)\)不断减小,直到为0。
\includegraphics{attachment:\%E6\%A2\%AF\%E5\%BA\%A6\%E4\%B8\%8B\%E9\%99\%8D\%E6\%B3\%95.png}

    \begin{figure}
\centering
\includegraphics{attachment:\%E6\%A2\%AF\%E5\%BA\%A6\%E4\%B8\%8B\%E9\%99\%8D\%E6\%B3\%95-udacity.png}
\caption{\%E6\%A2\%AF\%E5\%BA\%A6\%E4\%B8\%8B\%E9\%99\%8D\%E6\%B3\%95-udacity.png}
\end{figure}

    \hypertarget{ux4ee3ux7801ux5b9eux73b0}{%
\subsubsection{代码实现}\label{ux4ee3ux7801ux5b9eux73b0}}

    \begin{Verbatim}[commandchars=\\\{\}]
{\color{incolor}In [{\color{incolor}15}]:} \PY{n}{X} \PY{o}{=} \PY{n}{np}\PY{o}{.}\PY{n}{array}\PY{p}{(}\PY{p}{[}
             \PY{p}{[}\PY{l+m+mf}{3.393533211}\PY{p}{,} \PY{l+m+mf}{2.331273381}\PY{p}{]}\PY{p}{,}
             \PY{p}{[}\PY{l+m+mf}{3.110073483}\PY{p}{,} \PY{l+m+mf}{1.781539638}\PY{p}{]}\PY{p}{,}
             \PY{p}{[}\PY{l+m+mf}{1.343808831}\PY{p}{,} \PY{l+m+mf}{3.368360954}\PY{p}{]}\PY{p}{,}
             \PY{p}{[}\PY{l+m+mf}{3.582294042}\PY{p}{,} \PY{l+m+mf}{4.679179110}\PY{p}{]}\PY{p}{,}
             \PY{p}{[}\PY{l+m+mf}{2.280362439}\PY{p}{,} \PY{l+m+mf}{2.866990263}\PY{p}{]}\PY{p}{,}
             \PY{p}{[}\PY{l+m+mf}{7.423436942}\PY{p}{,} \PY{l+m+mf}{4.696522875}\PY{p}{]}\PY{p}{,}
             \PY{p}{[}\PY{l+m+mf}{5.745051997}\PY{p}{,} \PY{l+m+mf}{3.533989803}\PY{p}{]}\PY{p}{,}
             \PY{p}{[}\PY{l+m+mf}{9.172168622}\PY{p}{,} \PY{l+m+mf}{2.511101045}\PY{p}{]}\PY{p}{,}
             \PY{p}{[}\PY{l+m+mf}{7.792783481}\PY{p}{,} \PY{l+m+mf}{3.424088941}\PY{p}{]}\PY{p}{,}
             \PY{p}{[}\PY{l+m+mf}{7.939820817}\PY{p}{,} \PY{l+m+mf}{0.791637231}\PY{p}{]}
         \PY{p}{]}\PY{p}{)}
         \PY{n}{y} \PY{o}{=} \PY{n}{np}\PY{o}{.}\PY{n}{array}\PY{p}{(}\PY{p}{[}\PY{o}{\PYZhy{}}\PY{l+m+mi}{1}\PY{p}{,} \PY{o}{\PYZhy{}}\PY{l+m+mi}{1}\PY{p}{,} \PY{o}{\PYZhy{}}\PY{l+m+mi}{1}\PY{p}{,} \PY{o}{\PYZhy{}}\PY{l+m+mi}{1}\PY{p}{,} \PY{o}{\PYZhy{}}\PY{l+m+mi}{1}\PY{p}{,}  \PY{l+m+mi}{1}\PY{p}{,}  \PY{l+m+mi}{1}\PY{p}{,}  \PY{l+m+mi}{1}\PY{p}{,}  \PY{l+m+mi}{1}\PY{p}{,}  \PY{l+m+mi}{1}\PY{p}{]}\PY{p}{)}
\end{Verbatim}


    \begin{Verbatim}[commandchars=\\\{\}]
{\color{incolor}In [{\color{incolor}16}]:} \PY{n}{plt}\PY{o}{.}\PY{n}{scatter}\PY{p}{(}\PY{n}{X}\PY{p}{[}\PY{p}{:}\PY{p}{,} \PY{l+m+mi}{0}\PY{p}{]}\PY{p}{,} \PY{n}{X}\PY{p}{[}\PY{p}{:}\PY{p}{,} \PY{l+m+mi}{1}\PY{p}{]}\PY{p}{,} \PY{n}{marker}\PY{o}{=}\PY{l+s+s1}{\PYZsq{}}\PY{l+s+s1}{o}\PY{l+s+s1}{\PYZsq{}}\PY{p}{,} \PY{n}{c}\PY{o}{=}\PY{n}{y}\PY{p}{)}
\end{Verbatim}


\begin{Verbatim}[commandchars=\\\{\}]
{\color{outcolor}Out[{\color{outcolor}16}]:} <matplotlib.collections.PathCollection at 0x2c7376c6e48>
\end{Verbatim}
            
    \begin{center}
    \adjustimage{max size={0.9\linewidth}{0.9\paperheight}}{output_21_1.png}
    \end{center}
    { \hspace*{\fill} \\}
    
    \begin{Verbatim}[commandchars=\\\{\}]
{\color{incolor}In [{\color{incolor}17}]:} \PY{c+c1}{\PYZsh{} 符号函数}
         \PY{k}{def} \PY{n+nf}{sign}\PY{p}{(}\PY{n}{x}\PY{p}{,} \PY{n}{w}\PY{p}{,} \PY{n}{b}\PY{p}{)}\PY{p}{:}
             \PY{k}{if} \PY{n}{x}\PY{o}{.}\PY{n}{dot}\PY{p}{(}\PY{n}{w}\PY{p}{)} \PY{o}{+} \PY{n}{b} \PY{o}{\PYZgt{}}\PY{o}{=} \PY{l+m+mi}{0} \PY{p}{:}
                 \PY{k}{return} \PY{l+m+mi}{1}
             \PY{k}{else}\PY{p}{:}
                 \PY{k}{return} \PY{o}{\PYZhy{}}\PY{l+m+mi}{1}
\end{Verbatim}


    \begin{Verbatim}[commandchars=\\\{\}]
{\color{incolor}In [{\color{incolor}18}]:} \PY{c+c1}{\PYZsh{} 损失函数}
         \PY{k}{def} \PY{n+nf}{loss\PYZus{}func}\PY{p}{(}\PY{n}{w}\PY{p}{,} \PY{n}{b}\PY{p}{)}\PY{p}{:}
             \PY{n}{loss} \PY{o}{=} \PY{l+m+mi}{0}
             \PY{k}{for} \PY{n}{xi}\PY{p}{,} \PY{n}{yi} \PY{o+ow}{in} \PY{n+nb}{zip}\PY{p}{(}\PY{n}{X}\PY{p}{,} \PY{n}{y}\PY{p}{)}\PY{p}{:}
                 \PY{n}{y\PYZus{}hat} \PY{o}{=} \PY{n}{sign}\PY{p}{(}\PY{n}{xi}\PY{p}{,} \PY{n}{w}\PY{p}{,} \PY{n}{b}\PY{p}{)}
                 \PY{k}{if} \PY{n}{yi} \PY{o}{!=} \PY{n}{y\PYZus{}hat}\PY{p}{:}
                     \PY{n}{loss} \PY{o}{+}\PY{o}{=} \PY{n}{yi} \PY{o}{*} \PY{p}{(}\PY{n}{xi}\PY{o}{.}\PY{n}{dot}\PY{p}{(}\PY{n}{w}\PY{p}{)} \PY{o}{+} \PY{n}{b}\PY{p}{)}
             \PY{k}{return} \PY{o}{\PYZhy{}}\PY{n}{loss}
\end{Verbatim}


    \begin{Verbatim}[commandchars=\\\{\}]
{\color{incolor}In [{\color{incolor}19}]:} \PY{k}{def} \PY{n+nf}{train}\PY{p}{(}\PY{n}{X}\PY{p}{,} \PY{n}{y}\PY{p}{,} \PY{n}{eta}\PY{o}{=}\PY{l+m+mf}{1e\PYZhy{}1}\PY{p}{,} \PY{n}{epoch}\PY{o}{=}\PY{l+m+mi}{100}\PY{p}{)}\PY{p}{:}
             \PY{n}{w\PYZus{}list} \PY{o}{=} \PY{p}{[}\PY{p}{]}
             \PY{n}{b\PYZus{}list} \PY{o}{=} \PY{p}{[}\PY{p}{]}
             \PY{n}{loss\PYZus{}list} \PY{o}{=} \PY{p}{[}\PY{p}{]}
             \PY{c+c1}{\PYZsh{} 随机生成w和b的初始值}
             \PY{n}{w} \PY{o}{=} \PY{n}{np}\PY{o}{.}\PY{n}{random}\PY{o}{.}\PY{n}{random}\PY{p}{(}\PY{n}{X}\PY{p}{[}\PY{l+m+mi}{0}\PY{p}{]}\PY{o}{.}\PY{n}{shape}\PY{p}{)}
             \PY{n}{b} \PY{o}{=} \PY{n}{np}\PY{o}{.}\PY{n}{random}\PY{o}{.}\PY{n}{random}\PY{p}{(}\PY{p}{)}
             \PY{c+c1}{\PYZsh{} epoch 代表了所有的样本要循环多少遍}
             \PY{k}{for} \PY{n}{e} \PY{o+ow}{in} \PY{n+nb}{range}\PY{p}{(}\PY{n}{epoch}\PY{p}{)}\PY{p}{:}
                 \PY{c+c1}{\PYZsh{} 通过损失函数计算loss}
                 \PY{n}{loss} \PY{o}{=} \PY{n}{loss\PYZus{}func}\PY{p}{(}\PY{n}{w}\PY{p}{,} \PY{n}{b}\PY{p}{)}
                 \PY{n+nb}{print}\PY{p}{(}\PY{l+s+s1}{\PYZsq{}}\PY{l+s+s1}{epoch }\PY{l+s+si}{\PYZob{}\PYZcb{}}\PY{l+s+s1}{: loos=}\PY{l+s+si}{\PYZob{}\PYZcb{}}\PY{l+s+s1}{\PYZsq{}}\PY{o}{.}\PY{n}{format}\PY{p}{(}\PY{n}{e}\PY{p}{,} \PY{n}{loss}\PY{p}{)}\PY{p}{)}
                 \PY{c+c1}{\PYZsh{} 把w b loss 存了起来,为了画图}
                 \PY{n}{w\PYZus{}list}\PY{o}{.}\PY{n}{append}\PY{p}{(}\PY{n+nb}{list}\PY{p}{(}\PY{n}{w}\PY{p}{)}\PY{p}{)}  \PY{c+c1}{\PYZsh{} 注意list(w)}
                 \PY{n}{b\PYZus{}list}\PY{o}{.}\PY{n}{append}\PY{p}{(}\PY{n}{b}\PY{p}{)}
                 \PY{n}{loss\PYZus{}list}\PY{o}{.}\PY{n}{append}\PY{p}{(}\PY{n}{loss}\PY{p}{)}
                 \PY{c+c1}{\PYZsh{} loss 为0的时候就可以结束计算了}
                 \PY{k}{if} \PY{n}{loss} \PY{o}{==} \PY{l+m+mi}{0}\PY{p}{:}
                     \PY{k}{break}
                 \PY{c+c1}{\PYZsh{} 对于预测错误的样本,更新w 和 b}
                 \PY{k}{for} \PY{n}{xi}\PY{p}{,} \PY{n}{yi} \PY{o+ow}{in} \PY{n+nb}{zip}\PY{p}{(}\PY{n}{X}\PY{p}{,} \PY{n}{y}\PY{p}{)}\PY{p}{:}
                     \PY{k}{if} \PY{n}{yi} \PY{o}{*} \PY{p}{(}\PY{n}{xi}\PY{o}{.}\PY{n}{dot}\PY{p}{(}\PY{n}{w}\PY{p}{)} \PY{o}{+} \PY{n}{b}\PY{p}{)} \PY{o}{\PYZlt{}} \PY{l+m+mi}{0}\PY{p}{:}
                         \PY{n}{w} \PY{o}{+}\PY{o}{=} \PY{n}{eta} \PY{o}{*} \PY{n}{yi} \PY{o}{*} \PY{n}{xi}
                         \PY{n}{b} \PY{o}{+}\PY{o}{=} \PY{n}{eta} \PY{o}{*} \PY{n}{yi}
         
             \PY{k}{return} \PY{n}{w\PYZus{}list}\PY{p}{,} \PY{n}{b\PYZus{}list}\PY{p}{,} \PY{n}{loss\PYZus{}list}
\end{Verbatim}


    \begin{Verbatim}[commandchars=\\\{\}]
{\color{incolor}In [{\color{incolor}20}]:} \PY{n}{w\PYZus{}list}\PY{p}{,} \PY{n}{b\PYZus{}list}\PY{p}{,} \PY{n}{loss\PYZus{}list} \PY{o}{=} \PY{n}{train}\PY{p}{(}\PY{n}{X}\PY{p}{,} \PY{n}{y}\PY{p}{,} \PY{n}{eta}\PY{o}{=}\PY{l+m+mf}{1e\PYZhy{}2}\PY{p}{)}
\end{Verbatim}


    \begin{Verbatim}[commandchars=\\\{\}]
epoch 0: loos=24.90798108299876
epoch 1: loos=20.520109858516005
epoch 2: loos=16.132238634033254
epoch 3: loos=11.744367409550499
epoch 4: loos=7.356496185067748
epoch 5: loos=5.999663324318092
epoch 6: loos=5.070175966038305
epoch 7: loos=2.455825909373437
epoch 8: loos=2.4544828030477035
epoch 9: loos=2.453139696721969
epoch 10: loos=2.7679437417010377
epoch 11: loos=1.397885088295024
epoch 12: loos=1.561710561834354
epoch 13: loos=0.9364797047068084
epoch 14: loos=1.0958897060109387
epoch 15: loos=1.2597151795502688
epoch 16: loos=1.423540653089599
epoch 17: loos=1.5873661266289294
epoch 18: loos=0.9577197972661837
epoch 19: loos=1.1215452708055138
epoch 20: loos=1.285370744344844
epoch 21: loos=1.4491962178841742
epoch 22: loos=1.6130216914235045
epoch 23: loos=0.9833753620607588
epoch 24: loos=1.1472008356000891
epoch 25: loos=1.3110263091394192
epoch 26: loos=1.4748517826787495
epoch 27: loos=1.6386772562180796
epoch 28: loos=1.009030926855334
epoch 29: loos=1.1728564003946642
epoch 30: loos=1.3366818739339945
epoch 31: loos=1.5005073474733246
epoch 32: loos=1.6643328210126551
epoch 33: loos=1.0346864916499092
epoch 34: loos=1.1985119651892395
epoch 35: loos=1.3623374387285698
epoch 36: loos=0.2556860362369003
epoch 37: loos=0.3195150727700192
epoch 38: loos=0.3833441093031381
epoch 39: loos=0.4713324575054806
epoch 40: loos=0.5628718622522432
epoch 41: loos=0.6583320560104728
epoch 42: loos=0.8001353145728162
epoch 43: loos=0.9419385731351597
epoch 44: loos=1.0837418316975032
epoch 45: loos=1.2270710063039871
epoch 46: loos=0.19802605543999505
epoch 47: loos=0.6358923480187293
epoch 48: loos=0.7776956065810728
epoch 49: loos=0.9194988651434164
epoch 50: loos=1.0613021237057598
epoch 51: loos=0.11675923337651209
epoch 52: loos=0.4966554982563221
epoch 53: loos=0.613452640026986
epoch 54: loos=0.7552558985893294
epoch 55: loos=0.8970591571516728
epoch 56: loos=0.0435242679202645
epoch 57: loos=0.43590441604210667
epoch 58: loos=0.5483564097317708
epoch 59: loos=0.6901596682941143
epoch 60: loos=0.8319629268564582
epoch 61: loos=0.021085538754870688
epoch 62: loos=0.39568222077556586
epoch 63: loos=0.4832601794365557
epoch 64: loos=0.6250634379988995
epoch 65: loos=0.7668666965612434
epoch 66: loos=0

    \end{Verbatim}

    \begin{Verbatim}[commandchars=\\\{\}]
{\color{incolor}In [{\color{incolor}22}]:} \PY{c+c1}{\PYZsh{} 生成直线数据}
         \PY{n}{line\PYZus{}x0} \PY{o}{=} \PY{p}{[}\PY{l+m+mi}{1}\PY{p}{,} \PY{l+m+mi}{9}\PY{p}{]}
         \PY{n}{line\PYZus{}x1\PYZus{}list} \PY{o}{=} \PY{p}{[}\PY{p}{(}\PY{p}{(}\PY{o}{\PYZhy{}}\PY{n}{bi}\PY{o}{\PYZhy{}}\PY{n}{wi0}\PY{o}{*}\PY{p}{(}\PY{l+m+mi}{9}\PY{p}{)}\PY{p}{)}\PY{o}{/} \PY{n}{wi1}\PY{p}{,} \PY{p}{(}\PY{l+m+mi}{1}\PY{o}{*}\PY{n}{wi0}\PY{o}{\PYZhy{}}\PY{n}{bi}\PY{p}{)} \PY{o}{/} \PY{n}{wi1}\PY{p}{)} \PY{k}{for} \PY{p}{(}\PY{n}{wi0}\PY{p}{,} \PY{n}{wi1}\PY{p}{)}\PY{p}{,} \PY{n}{bi} \PY{o+ow}{in} \PY{n+nb}{zip}\PY{p}{(}\PY{n}{w\PYZus{}list}\PY{p}{,} \PY{n}{b\PYZus{}list}\PY{p}{)}\PY{p}{]}
\end{Verbatim}


    \begin{Verbatim}[commandchars=\\\{\}]
{\color{incolor}In [{\color{incolor}28}]:} \PY{c+c1}{\PYZsh{} 画图代码}
         \PY{k+kn}{from} \PY{n+nn}{matplotlib}\PY{n+nn}{.}\PY{n+nn}{animation} \PY{k}{import} \PY{n}{FuncAnimation}
         \PY{o}{\PYZpc{}}\PY{k}{matplotlib} notebook
         
         \PY{n}{fig}\PY{p}{,} \PY{n}{ax} \PY{o}{=} \PY{n}{plt}\PY{o}{.}\PY{n}{subplots}\PY{p}{(}\PY{p}{)}
         \PY{n}{ax}\PY{o}{.}\PY{n}{set\PYZus{}ylim}\PY{p}{(}\PY{o}{\PYZhy{}}\PY{l+m+mi}{5}\PY{p}{,} \PY{l+m+mi}{10}\PY{p}{)}
         \PY{n}{ax}\PY{o}{.}\PY{n}{set\PYZus{}xlim}\PY{p}{(}\PY{l+m+mi}{0}\PY{p}{,} \PY{l+m+mi}{10}\PY{p}{)}
         
         \PY{n}{plt}\PY{o}{.}\PY{n}{scatter}\PY{p}{(}\PY{n}{X}\PY{p}{[}\PY{p}{:}\PY{p}{,} \PY{l+m+mi}{0}\PY{p}{]}\PY{p}{,} \PY{n}{X}\PY{p}{[}\PY{p}{:}\PY{p}{,} \PY{l+m+mi}{1}\PY{p}{]}\PY{p}{,} \PY{n}{marker}\PY{o}{=}\PY{l+s+s1}{\PYZsq{}}\PY{l+s+s1}{o}\PY{l+s+s1}{\PYZsq{}}\PY{p}{,} \PY{n}{c}\PY{o}{=}\PY{n}{y}\PY{p}{)}
         \PY{n}{line\PYZus{}x1} \PY{o}{=} \PY{n}{line\PYZus{}x1\PYZus{}list}\PY{p}{[}\PY{l+m+mi}{0}\PY{p}{]}
         \PY{n}{line}\PY{p}{,}  \PY{o}{=} \PY{n}{ax}\PY{o}{.}\PY{n}{plot}\PY{p}{(}\PY{n}{line\PYZus{}x0}\PY{p}{,} \PY{n}{line\PYZus{}x1}\PY{p}{,} \PY{l+s+s1}{\PYZsq{}}\PY{l+s+s1}{r\PYZhy{}}\PY{l+s+s1}{\PYZsq{}}\PY{p}{,} \PY{n}{linewidth}\PY{o}{=}\PY{l+m+mi}{2}\PY{p}{)}
         
         \PY{k}{def} \PY{n+nf}{update}\PY{p}{(}\PY{n}{i}\PY{p}{)}\PY{p}{:}
             \PY{n}{label} \PY{o}{=} \PY{l+s+s1}{\PYZsq{}}\PY{l+s+s1}{epoch }\PY{l+s+si}{\PYZob{}\PYZcb{}}\PY{l+s+s1}{\PYZsq{}}\PY{o}{.}\PY{n}{format}\PY{p}{(}\PY{n}{i}\PY{p}{)}
             
             \PY{n}{line}\PY{o}{.}\PY{n}{set\PYZus{}ydata}\PY{p}{(}\PY{n}{line\PYZus{}x1\PYZus{}list}\PY{p}{[}\PY{n}{i}\PY{p}{]}\PY{p}{)}
             \PY{n}{ax}\PY{o}{.}\PY{n}{set\PYZus{}xlabel}\PY{p}{(}\PY{n}{label}\PY{p}{)}
             \PY{k}{return} \PY{n}{line}\PY{p}{,} \PY{n}{ax}
         
         \PY{n}{anim} \PY{o}{=} \PY{n}{FuncAnimation}\PY{p}{(}\PY{n}{fig}\PY{p}{,} \PY{n}{update}\PY{p}{,} \PY{n}{frames}\PY{o}{=}\PY{n}{np}\PY{o}{.}\PY{n}{arange}\PY{p}{(}\PY{l+m+mi}{0}\PY{p}{,} \PY{n+nb}{len}\PY{p}{(}\PY{n}{loss\PYZus{}list}\PY{p}{)}\PY{p}{)}\PY{p}{,} \PY{n}{interval}\PY{o}{=}\PY{l+m+mi}{500}\PY{p}{)}
         
         \PY{n}{plt}\PY{o}{.}\PY{n}{show}\PY{p}{(}\PY{p}{)}
\end{Verbatim}


    
    \begin{verbatim}
<IPython.core.display.Javascript object>
    \end{verbatim}

    
    
    \begin{verbatim}
<IPython.core.display.HTML object>
    \end{verbatim}

    

    % Add a bibliography block to the postdoc
    
    
    
    \end{document}
